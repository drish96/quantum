\documentclass[11pt]{article}
\usepackage{graphicx, amsmath, amssymb, amsthm, mdframed, enumerate, tikz, tensor}

\newenvironment{solution}{\begin{mdframed}[skipabove=\baselineskip,innertopmargin=\baselineskip,innerbottommargin=\baselineskip]
  %\ignorespaces
  }{\end{mdframed}}


\begin{document}
%\begin{bmatrix}\end{bmatrix}
\author{Drishan Sarkar}
\title{CSCI-PHYS 3090 -- Quantum Computing\\ Problem Set \#2 \\  Due: Wednesday, February 3, at 11am}
\maketitle

 
 \noindent 1. [No communication between independent systems] (/40)

... show that the independent evolution of two separate systems
does not communicate information, which seems intuitive. We do this by showing that the
unitary evolution of each system commutes, that is, we can evolve the first system and then
the second, or the other way around, it makes no difference.
\begin{enumerate}[(i)]
\item Let $X$ and $Y$ be $n$ and $m$ dimensional Hilbert spaces, respectively. What is the
dimension of the product space $X \otimes Y$? How could you write down its basis vectors
in terms of the basis vectors of $X$ and $Y$ ?\\
\item Let $U$ be a unitary operator acting on $X$, and $V$ a unitary operator acting on $Y$. Find
the action of $I \otimes V$ on an arbitrary basis vector of the product space, where $I$ is the
identity matrix of appropriate dimension.\\
\item Now find the action of $(U\otimes I)(I\otimes V)$ on the same arbitrary basis vector of the product
space.
\item Show that $(U\otimes I)(I\otimes V) = (I\otimes V)(U\otimes I)$ for all such $U,V$.
\end{enumerate}

\begin{solution}
\begin{enumerate}[(i)]
    \item
    The product space $X \otimes Y$ is $nm$-dimensional. Let the basis vectors of $X$ be $\{\overrightarrow{e}_1,...,\overrightarrow{e}_n\}$ and the basis vectors of $Y$ be $\{\overrightarrow{f}_1,...,\overrightarrow{f}_m\}$. We could write down the basis vectors of the product space as $\overrightarrow{e}_i\otimes \overrightarrow{f}_j$ where $i=1,...,n$ and $j=1,...,m$.
    \item 
    $I\otimes V = \begin{bmatrix}
    V & 0 & \cdots & 0 \\
    0 & V & \cdots & 0 \\
    \vdots & \vdots & \ddots & \vdots \\
    0 & 0 & \cdots & V 
    \end{bmatrix}$
    \item 
    $(U\otimes I)(I\otimes V) = 
    \begin{bmatrix}
    u_{11}I & u_{12}I & \cdots & u_{1n}I \\
    u_{12}I & u_{22}I & \cdots & u_{2n}I \\
    \vdots & \vdots & \ddots & \vdots \\
    u_{n1}I & u_{n2}I & \cdots & u_{nn}I 
    \end{bmatrix}\begin{bmatrix}
    V & 0 & \cdots & 0 \\
    0 & V & \cdots & 0 \\
    \vdots & \vdots & \ddots & \vdots \\
    0 & 0 & \cdots & V 
    \end{bmatrix}\\ = \begin{bmatrix}
    Vu_{11}I & Vu_{12}I & \cdots & Vu_{1n}I \\
    Vu_{12}I & Vu_{22}I & \cdots & Vu_{2n}I \\
    \vdots & \vdots & \ddots & \vdots \\
    Vu_{n1}I & Vu_{n2}I & \cdots & Vu_{nn}I 
    \end{bmatrix}$
    \item 
    $(I\otimes V)(U\otimes I) = 
    \begin{bmatrix}
    V & 0 & \cdots & 0 \\
    0 & V & \cdots & 0 \\
    \vdots & \vdots & \ddots & \vdots \\
    0 & 0 & \cdots & V 
    \end{bmatrix}\begin{bmatrix}
    u_{11}I & u_{12}I & \cdots & u_{1n}I \\
    u_{12}I & u_{22}I & \cdots & u_{2n}I \\
    \vdots & \vdots & \ddots & \vdots \\
    u_{n1}I & u_{n2}I & \cdots & u_{nn}I 
    \end{bmatrix} \\ = \begin{bmatrix}
    Vu_{11}I & Vu_{12}I & \cdots & Vu_{1n}I \\
    Vu_{12}I & Vu_{22}I & \cdots & Vu_{2n}I \\
    \vdots & \vdots & \ddots & \vdots \\
    Vu_{n1}I & Vu_{n2}I & \cdots & Vu_{nn}I 
    \end{bmatrix}\vspace{2em}$ \\ This is the same as the result from part (iii), showing for all such $U,V$ that $(U\otimes I)(I\otimes V) = (I\otimes V)(U\otimes I)$.
\end{enumerate}
\end{solution}

\newpage

 \noindent 2. [Tensor Products] (/30)
 
 This problem is to give you more practice working with tensor products. In this case, we
will consider tensor products of both operators and Hilbert spaces, representing physically
both separated systems and separated operations on those systems.
\begin{enumerate}[(i)]
\item Let $X$ and $Y$ be $n$ and $m$ dimensional Hilbert spaces, respectively. Let $|x\rangle \in X$,
$|y\rangle \in Y$. Let $U$ be a unitary operator acting on $X$ and $V$ a unitary operator acting
on $Y$. What is the dimension of the tensor product operator $U\otimes V$? What is the dimension of that operator acting on a ket, i.e. $(U \otimes V )(|x\otimes y\rangle)$?
\item Show that $$(U\otimes V)(|x \otimes y \rangle) = (U|x\rangle)\otimes(V|y\rangle)$$ for all such $U,V$.
\end{enumerate}

\begin{solution}
\begin{enumerate}[(i)]
    \item 
    The tensor product operator $U\otimes V$ would have dimension $nm\times nm$, and that operator acting on a ket would be $nm\times 1$.
    \item  Let the basis vectors of $X$ be $\{\overrightarrow{e}_1,...,\overrightarrow{e}_n\}$ and the basis vectors of $Y$ be $\{\overrightarrow{f}_1,...,\overrightarrow{f}_m\}$. Then, \begin{align*}
        (U\otimes V)(|x \otimes y \rangle) &= U\{\overrightarrow{e}_1,...,\overrightarrow{e}_n\} \otimes V\{\overrightarrow{f}_1,...,\overrightarrow{f}_m\}\\
        &= (U|x\rangle)\otimes(V|y\rangle)
    \end{align*}
\end{enumerate}
\end{solution}

\newpage

 \noindent 3. [Bernoulli Vectors] (/30)
 
The multivariate Bernoulli vectors are random unit vectors that describe a random experiment
with $k$ possible outcomes, each outcome occurring with probability $p_k$. We will consider
two Boolean random variables in this context occurring in two independent spaces.

\begin{enumerate}[(i)]
\item Imagine a Boolean random variable $P$ that is 1 with probability $p$ and 0 with probability $1 - p$. We can write its distribution as the two-dimensional vector $[1 - p, p]^T \in \textbf{R}^2$. Assume now that we have two such independent random variables $P$ and $Q$, with probability $p$ and $q$ of being 1, respectively. Write down the four-dimensional vector giving the probabilities of the four possible values of the pair $(P,Q)$, using the order 00, 01, 10, and 11.

\item Write the same four-dimensional vector in terms of the vectors $|u\rangle = [1-p,p]^T$ and $|v\rangle = [1-q,q]^T$.

\item Find a $2\times2$ matrix $X$ such that if $|u\rangle = [1-p, p]^T$ is the distribution of the Boolean random variable $P$, then $X|u\rangle$ is the distribution of the random variable $P$ obtained by negating the bit $P$.

\item Let $R$ be an arbitrary random variable taking values in $\{00, 01, 10, 11\}$, and let $r \in \textbf{R}^4$ be its distribution. Find a $4\times4$ matrix $\tilde{X}$ such that $\tilde{X}r$ is the distribution of neg$_1(R)$ where neg$_1$ negates the first bit of $R$.

\end{enumerate}


\begin{solution}
\begin{enumerate}[(i)]
    \item $\begin{bmatrix}(1-p)(1-p),&(1-p)q,&p(1-q),&pq\end{bmatrix}^T$
    
    \item $|u\otimes v\rangle$
    
    \item $X = \begin{bmatrix}0&1\\1&0\end{bmatrix}$
    
    \item $\tilde{X} = \begin{bmatrix}
    0&0&1&0\\0&0&0&1\\1&0&0&0\\0&1&0&0
    \end{bmatrix}$
\end{enumerate}
\end{solution}


 
 
 \end{document}
